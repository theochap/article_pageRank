
\subsection{Theoretical proofs of convergence speed}
\subsubsection{Markov-Chain Mixing-Time and Convergence speed}
\label{section:theoretical_proofs_of_convergence}
\begin{definition}[Markov-Chain Mixing-Time \cite{levin_peres_wilmer_propp_wilson_2017}, section 4.5]
Let P be the transition matrix of a given regular Markov chain C, whose associated graph is G=(V,E). Then, $(P^t(x, .))_{t\in \mathbb{N}}$, with $x \in V$ converges to the probability limit distribution $\pi$ (see theorem \ref{theo:limit_distrib_markov_chain}).

One defines the distance to the probability limit distribution by :
\begin{equation}
    d(t) := \underset{x \in V}{max} \ \underset{y \in V}{max} \ \| P^t(x, y) - \pi(y) \|
\end{equation}

Then, the mixing time of the Markov Chain C is defined as : 
\begin{equation}
    t_{mix}(\epsilon) := min\{t : d(t) \leq \epsilon \}
\end{equation}

\end{definition}

Intuitively, the mixing-time of C is the minimal number of steps a random walker starting from any node needs to perform to so that its presence probability distribution becomes $\epsilon$-close to the limit probability distribution $\pi$.

We will now state a fundamental theorem to link the speed of convergence of random walkers propagating on Markov-Chains to the underlying graph structure.

\begin{theo}[Markov-Chain Mixing-Time spectral-gap and Cheeger's constant bounds \cite{levin_peres_wilmer_propp_wilson_2017}]
Let G be a strongly connected graph and let $\mathcal{L}$ be the Laplacian of G,

\end{theo}